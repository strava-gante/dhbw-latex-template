%
%   Tom Richter
%   auf Basis einer Vorlage von Dr. Julian Reichwald und Prof. Dr. Jörg Baumgart
%
%	ACHTUNG: Für das Erstellen des Literaturverzeichnisses wird das modernere Paket biblatex
%			 in Kombination mit biber verwendet -- nicht mehr das ältere BibTex!
% 			 Bitte stellen Sie ggf. Ihre TeX-Umgebung
% 			 entsprechend ein (z.B. TeXStudio: Einstellungen --> Erzeugen --> Standard Bibliographieprogramm: biber)
%

\documentclass[
	12pt, % defaut=12
	BCOR=5mm,F
	DIV=12, % ca 3.5cm margins % default=12
	headinclude=on,
	footinclude=off,
	parskip=half,
	bibliography=totoc,
	listof=entryprefix,
	toc=listof,
	number=noenddot,
	plainfootsepline]{scrreprt}
	
%	Konfigurationsdateien einbinden
% !TEX root = ../main.tex

%-------------------
% 		HYPERREF
%-------------------

\usepackage{hyperref}
\hypersetup{
    colorlinks = true,
    % allcolors = [rgb]{0,0,0} % black
    allcolors = [rgb]{0,0,0.4} % blue
}

% URLs im Fließtext
\newcommand{\urlinline}[1]{\textsf{\footnotesize{\url{{#1}}}}}


%		TITLING
%-----------------------------------
\makeatletter
\newcommand*{\thetitle}{\@title}
\newcommand*{\theauthor}{\@author}
\newcommand*{\thedate}{\@date}
\newcommand*{\place}[1]{\newcommand{\theplace}{#1}}
\newcommand*{\thesis}[1]{\newcommand{\thethesis}{#1}}
\newcommand*{\university}[1]{\newcommand{\theuniversity}{#1}}
\newcommand*{\courseofstudies}[1]{\newcommand{\thecourseofstudies}{#1}}
\newcommand*{\timeperiod}[1]{\newcommand{\thetimeperiod}{#1}}
\newcommand*{\studentnr}[1]{\newcommand{\thestudentnr}{#1}}
\newcommand*{\course}[1]{\newcommand{\thecourse}{#1}}
\newcommand*{\company}[1]{\newcommand{\thecompany}{#1}}
\newcommand*{\supervisor}[1]{\newcommand{\thesupervisor}{#1}}
\newcommand*{\scientificsupervisor}[1]{\newcommand{\thescientificsupervisor}{#1}}
\makeatother

% PDF settings
\hypersetup{pdftitle={\thetitle}}
\hypersetup{pdfauthor={\theauthor}}


%-------------------
% 		List Styles
%-------------------
\usepackage{enumitem}
\usepackage{lineno}


%-------------------------
%		SCHRIFTART
%-------------------------
% Entweder Latin Modern oder Times / Helvetica
\usepackage{lmodern} %Latin modern font
%\usepackage{mathptmx}  %Helvetica / Times New Roman fonts (2 lines)
%\usepackage[scaled=.92]{helvet} %Helvetica / Times New Roman fonts (2 lines)


%-----------------------------------
%		SCHRIFT UND ENCODING
%-----------------------------------
\usepackage[T1]{fontenc}

\usepackage{eurosym} % euro symbol
\DeclareRobustCommand{\officialeuro}{%
  \ifmmode\expandafter\text\fi
  {\fontencoding{U}\fontfamily{eurosym}\selectfont e}}
  
\usepackage[utf8]{inputenc}
\DeclareUnicodeCharacter{20AC}{\euro} % Euro symbol character

\usepackage{setspace}

% Common abbreviation's
\usepackage{xspace}
\AtBeginDocument{
    \newcommand*{\eg}{e.g.\@\xspace}
    \newcommand*{\ie}{i.e.\@\xspace}
    \newcommand*{\Eg}{E.g.\@\xspace}
    \newcommand*{\Ie}{I.e.\@\xspace}
    \newcommand*{\cf}{cf.\@\xspace}
    \newcommand*{\Cf}{Cf.\@\xspace}
}


%---------------------------
%		BERECHNUNGEN
%---------------------------
\usepackage{calc} % Used for extra space below footsepline


%-------------------------
%		TABLES
%-------------------------
\usepackage[table,xcdraw]{xcolor}


%---------------------------------
%		SPRACHEINSTELLUNGEN
%---------------------------------
% Voreinstellungen für Deutsch und Englisch. Die nicht verwendete Sprache ist auszukommentieren.
% DEUTSCH
%\usepackage[ngerman]{babel}
%\usepackage[german=quotes]{csquotes}

% ENGLISH
%\usepackage[ngerman]{babel}
%\usepackage[german=quotes]{csquotes}

%ENGLISH & DEUTSCH (default english)
\usepackage[english, ngerman]{babel}
\usepackage[style=american]{csquotes} % Richtiges Setzen der Anführungszeichen mit \enquote{}
\AtBeginDocument{
    \selectlanguage{english}
}


%----------------------------
%		BIBLIOGRAFIE
%----------------------------
\AtBeginDocument{
    \setcounter{biburllcpenalty}{7000} % break urls
    \setcounter{biburlucpenalty}{8000} % break urls
}
% Voreinstellungen für Fußnotenzitate (Autor-Jahr), IEEE-Standard, Alphabetic-Stil und Havard-Stil. Die nicht verwendeten Stile müssen auskommentiert werden

%\usepackage[backend=biber, autocite=footnote, style=authoryear, dashed=false]{biblatex}                % Fußnotenzitate
% \usepackage[backend=biber, autocite=inline, style=ieee]{biblatex}                                     % IEEE-Stil
% \usepackage[backend=biber, autocite=inline, style=alphabetic]{biblatex}                               % Alphabetic-Stil
% \usepackage[backend=biber, autocite=inline, style=authoryear, dashed=false]{biblatex}		            % Harvard-Stil
\usepackage[backend=biber, autocite=inline, bibstyle=authoryear, citestyle=apa, dashed=false]{biblatex} % APA-Stil


% Fußnotenzitate mit YYYY-MM-DD in Bibliographie
% \usepackage[backend=biber, autocite=footnote, style=authoryear, dashed=false, urldate=edtf, date=edtf, seconds=true]{biblatex}

\DefineBibliographyStrings{ngerman}{  %Change u.a. to et al. (german only!)
	andothers = {{et\,al\adddot}},
}

\setlength{\bibparsep}{\parskip}		%add some space between biblatex entries in the bibliography
\addbibresource{adds/bibliography.bib}	%Add file bibliography.bib as biblatex resource


%----------------------
%		ACRONYME
%----------------------
%%%
%%% WICHTIG: Installieren Sie das neueste Acronyms-Paket!!!
%%%
\makeatletter
\usepackage[]{acronym}
\@ifpackagelater{acronym}{2015/03/20}
  {%
    \renewcommand*{\aclabelfont}[1]{\textbf{\textsf{\acsfont{#1}}}}
  }%
  {%
  }%
\makeatother


%--------------------
%		GLOSSAR
%--------------------
%\usepackage[toc]{glossaries}					% für Seitenreferenzen im Glossar
\usepackage[toc, nonumberlist]{glossaries}		% ohne Seitenreferenzen im Glossar


%--------------------
%		NUMBERING FIGURES/TABLES/FOOTNOTES/TOC/CHAPTERS
%--------------------
\usepackage{chngcntr}
\counterwithout{footnote}{chapter} % Zum Zählen der Fußnoten über Kapitel hinaus
% \counterwithin{figure}{section} % Zum Zählen der Bilder innerhalb von Sections
% \counterwithin{table}{section} % Zum Zählen der Tabellen innerhalb von Sections
\setcounter{tocdepth}{2} % TODO adapt if toc too long
\setcounter{secnumdepth}{2} % Numbering for subsubsections


%-----------------------------------
%		REFERENCES & CITING
%-----------------------------------
% \usepackage{letltxmacro}
% \AtBeginDocument{
%     \LetLtxMacro{\originalref}{\ref}
%     \renewcommand{\ref}[1]{(\cf~\originalref{#1}, p.~\pageref{#1})}
%     \newcommand{\imgref}[1]{(\cf~figure~\originalref{#1}, p.~\pageref{#1})}
% }


%-------------------------------
%		ZUSÄTZLICHE PAKETE
%-------------------------------
\usepackage{lipsum}				 % Blindtext
\setlipsumdefault{1-4}

\usepackage[activate]{microtype} % fix overfull hbox
\usepackage[pdftex]{graphicx}    % verschiene Bildformate einbinden
\usepackage{pdfpages}            % PDF einbinden
\usepackage[format=hang, justification=raggedright, margin=1cm, singlelinecheck=false]{caption}			% schönere Überschriften

\usepackage{booktabs}			% bessere Tabs
\usepackage{array}

% bessere Tabellen
\usepackage{multirow}
\setlength{\tabcolsep}{10pt}
\renewcommand{\arraystretch}{1.5}
\newcolumntype{P}[1]{>{\raggedright\arraybackslash}p{#1}}

% resource folder einbinden
\graphicspath{ {resources/} }


%--------------------------
%		Tikz diagram library
%--------------------------
\usepackage{tikz}
\usetikzlibrary{arrows,decorations.pathmorphing,backgrounds,fit,positioning,shapes.symbols,chains}

%--------------------------
%		TpX used packages
%--------------------------
\usepackage{color}
\DeclareGraphicsExtensions{.pdf,.png,.jpg,.jpeg,.mps}
\usepackage{pgf}
\usepackage{epic,bez123}
\usepackage{floatflt}% package for floatingfigure environment
\usepackage{wrapfig}% package for wrapfigure environment

%-------------------------
%		ALGORITHMEN
%-------------------------
\usepackage{algorithm}
\usepackage{algpseudocode}
\renewcommand{\listalgorithmname}{List of Algorithms}
\floatname{algorithm}{algorithm}

%------------------------------------
%		KOPFZEILE / FUßZEILE
%------------------------------------
%	   ACHTUNG! Einige einstellungen werden in master.tex erneut verändert
\RequirePackage[automark,headsepline,footsepline]{scrpage2}
\pagestyle{scrheadings}
\renewcommand*{\pnumfont}{\upshape\sffamily}
\renewcommand*{\headfont}{\upshape\sffamily}
\renewcommand*{\footfont}{\upshape\sffamily}
\renewcommand{\chaptermarkformat}{}

\clearscrheadfoot

% using cfoot centers the header
% use ofoot for outer side, especially with documentclass[twoside]
\cfoot[\rule{0pt}{\ht\strutbox+\dp\strutbox}\pagemark]{\rule{0pt}{\ht\strutbox+\dp\strutbox}\pagemark}

\ohead{\headmark}

%-----------------------------------
%		Fix Headheight
%-----------------------------------
\setlength{\headheight}{1.1\baselineskip}


% !TEX root = ../main.tex

%---------------------
%		LISTINGS
%---------------------
\usepackage{listings}
\usepackage{dirtree}
% Listings formatieren
\renewcommand{\lstlistlistingname}{List of Listings}
	
\lstdefinelanguage{Typescript} {
    keywords={
        break, const, continue, delete, do, get, set, while, for, in, of, function, async, await, if, else, in, instanceOf, label, let, new, return, try, catch, then, typeof, var, void, with, yield, string, true, false
    },
    ndkeywords={class, interface, export, extends, throw, implements, import, from, as, this, switch, case, public, private},
    sensitive=false,
    morecomment=[l]{//},
    morecomment=[s]{/*}{*/},
    morestring=[b]",
    morestring=[d]',
    morestring=[d]`
}

\lstdefinelanguage{json}{
    string=[s]{"}{"},
    comment=[l]{:}
}

\lstdefinelanguage{reqTimeline}{
    comment=[l]{|}
}

\definecolor{LinkColor}{HTML}{00007A}
\definecolor{CadetBlue}{HTML}{5f9ea0}
\definecolor{OliveGreen}{HTML}{556b2f}
\definecolor{Gray}{HTML}{808080}
\definecolor{LightGray}{HTML}{d3d3d3}

\lstset{
    basicstyle=\footnotesize\ttfamily,
    numbers=left,
	stepnumber=1,
	numbersep=5pt,
	numberstyle=\tiny,
	tabsize=2,
	showspaces=false,
	extendedchars=true,
	literate={€}{{\euro}}1,
	captionpos=b,
    showstringspaces=false,
    breaklines=true,
	breakautoindent=true,
	postbreak=\space,		
    keywordstyle=\ttfamily\bfseries\color{CadetBlue},
    ndkeywordstyle=\ttfamily\bfseries\color{LinkColor},
    identifierstyle=\ttfamily,
    stringstyle=\ttfamily\color{OliveGreen},
    commentstyle=\color{Gray},
    frame=tb,
    framesep=5pt,
    rulecolor=\color{LightGray},
    aboveskip=1cm,
    % belowskip=1cm
}
%   Settings laden
% !TEX root = ../main.tex

%-------------------
% 		Settings
%-------------------
\title{\LaTeX~ DHBW Template}
\author{Tom Richter}
\date{10 September 2018}
\place{Stuttgart}
\thesis{Bachelor Thesis}
\university{Cooperative State University Stuttgart Baden-Wuerttemberg}
\courseofstudies{Applied Computer Science}
\timeperiod{12 Weeks}
\studentnr{123456}
\course{TINF15A}
\company{DXC Technology}
\supervisor{Maximilian Maier, Master of Arts}
\scientificsupervisor{Prof. Dr. Anna Müller}

%-----------------------------------
%		Glossary
%-----------------------------------
\makeglossaries
%!TEX root = ../main.tex

\newglossaryentry{glossar}
{
    name=Glossar,
    description={A glossary is a list of terms in a particular domain of knowledge with definitions for those terms.}
}
	
\begin{document}

\singlespacing

%   Titlepage
% !TEX root = ../main.tex

\begin{titlepage}

\begin{minipage}{\textwidth}
		\vspace{-2cm}
		\noindent
		\raisebox{-0.5\height}{\includegraphics[width=7cm, keepaspectratio]{dxc_logo}}
		\hfill
		\raisebox{-0.5\height}{\includegraphics[width=5cm, keepaspectratio]{dhbw_logo}}
\end{minipage}

\enlargethispage{20mm}

\sffamily
\begin{center}
    \vspace*{24mm}  {\large\textbf{\thetitle}}       \\
    \vspace*{12mm}  {\large\textbf{\thethesis}}        \\
    \vspace*{24mm}   for the course of study         \\
    \vspace*{3mm}   {\large\textbf{\thecourseofstudies}} \\
    \vspace*{3mm}   at the                         \\
    \vspace*{3mm}   {\large\textbf{\theuniversity}}  \\
    \vspace*{12mm}  by                              \\
    \vspace*{3mm}   {\large\textbf{\theauthor}}      \\
    \vspace*{12mm}  \thedate \\

\vfill

\begin{minipage}{\textwidth}

\begin{tabbing}
	\hspace{1.85cm} \= \kill
	Project Period: \` \thetimeperiod \\[1.5mm]
	Student ID, Course: \` \thestudentnr, \thecourse\\[1.5mm]
	Company: \` \thecompany\\[1.5mm]
	Supervisor: \` \thesupervisor\\[1.5mm]
	Scientific Supervisor: \` \thescientificsupervisor\\[1.5mm]

\end{tabbing}
\end{minipage}

\end{center}

\end{titlepage}

\pagenumbering{Roman} % Römische Seitennummerierung
\normalfont

%----------------------------------------
% Verzeichnisse - nicht benötige Verzeichnisse bitte auskommentieren / löschen.
%----------------------------------------

\onehalfspacing
\pagestyle{scrheadings}
%   Sperrvermerk
%!TEX root = ../main.tex

\clearpage

\chapter*{Nondisclosure Statement}
\thispagestyle{scrheadings}

The \textit{Bachelor Thesis} on hand
\begin{center}{\itshape{} \thetitle\/}\end{center}
contains internal respectively confidential data of \textit{\thecompany}. It is intended solely for inspection by the assigned examiner, the \textit{Digital Service Innovation} department and, if necessary, the Audit Committee of the \theuniversity in \theplace. It is strictly forbidden
\begin{itemize}
    \item to distribute the content of this paper (including data, figures, tables, charts, etc.) as a whole or in extracts,
    \item to make copies or transcripts of this paper or of parts of it,
    \item to display this paper or make it available in digital, electronic or virtual form.
\end{itemize}
Exceptional cases may be considered through permission granted in written form by the author and the company.

\vspace{3em}

\theplace, \thedate
\vspace{4em}

\rule{6cm}{0.4pt}\\
\theauthor

\onehalfspacing
\pagestyle{scrheadings}

% Ehrenwörtliche Erklärung
% !TEX root = ../main.tex

\clearpage

\chapter*{Declaration of Authorship}
\thispagestyle{scrheadings}

I hereby declare:

\begin{itemize}
	\item that this thesis with the title \textit{\thetitle} is my own work and
	\item that I have not used any other sources or assistance than declared here.
	\item that I have not submitted the thesis as a whole or in part for a degree at any university
		or institution before.
	\item that I have not published this thesis before.
\end{itemize}

Furthermore, I confirm, that the presented electronic version of this thesis is identical to the printed version.\\
I am aware that an incorrect declaration will be followed by legal measures.

\vspace{3em}

\theplace, \thedate
\vspace{4em}

\rule{6cm}{0.4pt}\\
\theauthor


\singlespacing
\pagestyle{scrheadings}
%	Kurzfassung
% !TEX root = ../main.tex

\chapter*{Abstract}

\begingroup
  \begin{table}[h!]
    \setlength\tabcolsep{0pt}
    \begin{tabular}{p{3.5cm}p{10.0cm}}
      Title & \thetitle \\
      Author: & \theauthor \\
    \end{tabular}
  \end{table}
\endgroup

\hspace{2cm}

% Blindtext
\lipsum[1-2]

\onehalfspacing
\pagestyle{scrheadings}

% Vorangehende Notizen
% !TEX root = ../main.tex

\clearpage
\chapter*{Preliminary Notes}
\label{chap:prenotes}
\thispagestyle{scrheadings}

\paragraph{Gender Neutrality}
This paper is written in gender-neutral language. Any person is addressed using the singular \emph{they}. For example, instead of  \enquote{The user changes \emph{his} account settings.} this paper writes, \enquote{The user changes \emph{their} account settings.}

\paragraph{Trademarks}
Designations from third-party vendors are used in this paper. These designations may be trademarks or registered trademarks. Wherever the author is aware of such trademark, the designation will be written with an initial capital letter.

\paragraph{Formatting}
\textit{Italic text} is used to emphasize new technical terms, names of persons, product names when they are first used and to highlight general keywords. Within image captions, it is used to cite the origin of the figure.\\ 
\textbf{Bold text} is used to denote headings in general and subheadings of paragraphs.\\ 
\texttt{Monospaced text} is used to denote variables, function names and computer commands and reference them from listings.\\
Italic Symbols (\eg, $A$) are used in formulas and are used to refer to those symbols.


\singlespacing

%	Inhaltsverzeichnis
\begingroup
\renewcommand*{\chapterpagestyle}{scrheadings}
\pagestyle{scrheadings}
\tableofcontents
\clearpage
\endgroup

\pagestyle{scrheadings}

%	Abbildungsverzeichnis
\listoffigures

%	Tabellenverzeichnis
\listoftables

%	Listingsverzeichnis
\lstlistoflistings

% 	Algorithmenverzeichnis
\listofalgorithms

%   Symbolverzeichnis
% !TEX root = ../main.tex

\clearpage
\chapter*{List of Symbols}
\addcontentsline{toc}{chapter}{List of Symbols}

\begin{tabular}{ccl}
\textbf{Symbol} & \textbf{Unit} & \textbf{Description} \\ \hline
$l$ & $mm$ & Length \\
$r$ & $mm$ & Radius \\
$A$ & $mm^2$ & Area \\
$V$ & $mm^3$ & Volume \\
$t$ & $s$ & Time \\
$m$ & $g$ & Weight \\
$\vec{p}$ & $PSI$ & Pressure \\
$\vec{p}$ & $N$ & Pressure \\
$N.A.$ & $EUR$ & Price \\
\end{tabular}

% 	Abkürzungsverzeichnis (siehe Datei acronyms.tex!)
\input{adds/acronyms}

%----------------------------------------
% Start des Textteils der Arbeit
%----------------------------------------
\clearpage
\ihead{\chaptername~\thechapter} % Neue Header-Definition
\pagenumbering{arabic}  % Arabische Seitenzahlen

\onehalfspacing
\pagestyle{scrheadings}

%Einzelne Kapitel können hier eingefügt werden.
%Es ist vorgesehen alle Kapitel als eigene Dateien
% !TEX root = ../main.tex
\chapter{Gebrauchsanleitung}

\section{Übersicht über die Vorlage}
Die Vorlage wurde im UTF-8 Encoding erstellt. Sollten daher z.\,B. Umlaute in Ihrem \LaTeX-Editor nicht korrekt dargestellt werden, überprüfen Sie bitte die Encoding-Einstellungen des Editors. In seltenen Fällen müssen Sie die Vorlage danach noch einmal neu in den Editor einbinden.
Die Vorlage beinhaltet die folgenden, in Tabelle \ref{tab:dateien} aufgelisteten Dateien:

\begin{table}[H]
	\centering
    \begin{tabular}{lp{10cm}}
    	\textbf{Dateiname} & \textbf{Beschreibung}\\\toprule
    	\texttt{main.tex} & Die Hauptdatei. Alle anderen Dateien werden von dieser Datei eingezogen. \\
    	\texttt{abstract.tex} & Die Kurzfassung der Arbeit. \\
    	\texttt{acronyms.tex} & Definition von Abkürzungen. \\
    	\texttt{declarationonhonours.tex} & Ehrenwörtliche Erklärung.\\
    	\texttt{glossary.tex} & Glossar. \\
    	\texttt{nondisclosurenotice.tex} & Sperrvermerk.\\
    	\texttt{symbols.tex} & Definition von Symbolen.\\
    	\texttt{titlepage.tex} & Titelseite der Arbeit.\\
    	\texttt{code.tex} & Konfigurationseinstellungen von Code Listings.\\
    	\texttt{config.tex} & Konfigurationseinstellungen der einzelnen Pakete.\\
    	\texttt{settings.tex} & Persönliche Daten der Arbeit.\\
    	\texttt{01Introduction.tex} & Diese Anleitung.\\
    	\texttt{bibliography.bib} &  Die Literaturdatenbank -- hier können Sie die verwendete Literatur einpflegen.\\\bottomrule
    \end{tabular}
    \caption{Übersicht über die Dateien der Vorlage}
    \label{tab:dateien}
\end{table}

Es werden -- unter anderem -- die folgenden Zusatzpakete von dieser Vorlage eingezogen und sollten daher in aktuellen Versionen installiert sein:
\begin{itemize}
	\item\texttt{KOMA-Script} bzw. die Dokumentenklasse \texttt{scrreprt}
	\item\texttt{hyperref} für PDF-Informationen und Links
	\item \texttt{babel} für länderspezifische Einstellungen
	\item \texttt{csquotes} für sprachabhängige Anführungszeichen (Befehl: \texttt{\textbackslash enquote})
	\item \texttt{acronym} für das Erstellen des Abkürzungsverzeichnisses
	\item \texttt{booktabs} für das typografisch schöne Setzen von Tabellen
	\item \texttt{listings} für schöne Quelltexte
	\item \texttt{algorithm} für schöne Algorithmen
	\item \texttt{bibltatex} und \texttt{biber} für die Erstellung des Literaturverzeichnisses.
\end{itemize}
Alle Konfigurationen dieser Vorlage können in der Datei \texttt{config.tex} eingesehen und ggf. verändert werden. Bitte schauen Sie sich die entsprechenden Dokumentationen
der Pakete an (\url{https://www.ctan.org}), um deren Verwendung und Möglichkeiten jenseits der hier gezeigten Beispiele zu erlernen.


\section{Übersetzung von \LaTeX-Dateien}
Die Übersetzung von \LaTeX-Dateien erfolgt in mehreren Schritten und unter der Zuhilfenahme unterschiedlicher Programme. Das Hauptdokument (hier die Datei \texttt{main.tex}) wird mittels \texttt{pdflatex} zu einem PDF übersetzt. Ggf. ist eine mehrfache Übersetzung notwendig, um z.\,B. das Inhaltsverzeichnis korrekt darzustellen.

Für die Einbindung des Literaturverzeichnisses wird nicht mehr das ältere \texttt{bibtex}, sondern das neuere \texttt{biber} in Kombination mit \texttt{biblatex} verwendet. Bitte stellen Sie Ihren \LaTeX-Editor so ein, dass die Verwendung von Biber beim Übersetzungsprozess erfolgt.

\section{Verwendung von Akronymen}
Akronyme müssen in der Datei \texttt{acronyms.tex} definiert werden (schauen Sie sich hierzu bitte die entsprechende Paket-Dokumentation an!) Ein definiertes Akronym kann dann mit dem Befehl \texttt{\textbackslash ac} verwenden, so wird z.\,B. \texttt{\textbackslash ac\{DHBW\}} zu \ac{DHBW}. Im weiteren Verlauf wird das Acronym dann nur noch in der Kurzform dargestellt: \ac{DHBW}. Die Aufnahme eines verwendeten Akronyms in das Abkürzungsverzeichnis erfolgt automatisch.

\section{Zitieren von Quellen}
Mit dem Befehl \texttt{\textbackslash autocite} kann zitiert werden, z.\,B. so \autocite[Vgl.][S. 18ff.]{astm2012AM}. Sollen mehrere Referenzen auf einmal gesetzt werden, können Sie dies mit dem Befehl \texttt{\textbackslash autocites} erreichen, z.\,B. so\autocites[Vgl.][S. 10]{astm2012AM}[][S. 100]{reddy2016AM}. Wird \texttt{autocite} konsequent verwendet, kann in der Datei \texttt{config.tex} der Zitationsstil umgeschaltet werden, ohne dass im Text Veränderungen vorgenommen werden müssen. Vorkonfigurierte Stile sind Alphabetic, Harvard, Fußnotenzitat, IEEE, und APA-Style. Die Übernahme der Quellen in das Literaturverzeichnis erfolgt automatisch. Ein Beispiel für eine Online-Quelle ist ebenfalls enthalten.\autocite[Vgl.][]{3mfconsortium2018adoption}

Soll einer Abbildung eine Quellenangabe zugefügt werden, bietet es sich an, diese direkt in der jeweiligen Abbildungsbeschriftung zu hinterlegen. Hierfür kann der Befehl \texttt{\textbackslash cite} verwendet werden, um eine ungewollte Fußnote zu vermeiden. Ein Beispiel ist in Abbildung \ref{fig:test} zu sehen.


\section{Text in Anführungszeichen}
Soll ein Text in Anführungszeichen gesetzt werden, kann dies über den Befehl \texttt{\textbackslash enquote} \enquote{so erreicht werden}. Die Anführungszeichen ändern sich automatisch auf die jeweiligen Länderspezifika, wenn die Spracheinstellung des \texttt{babel}-Pakets geändert wird. Voreinstellung ist die amerikanische Verwendung von Anführungszeichen.


\section{Beispiele}
\lipsum[1]

\subsection{Unterabschnitte}
Es gibt neben \texttt{\textbackslash chapter} auch noch  \texttt{\textbackslash section}, \texttt{\textbackslash subsection}, \texttt{\textbackslash subsubsection} etc. Eine zu starke Untergliederung des Textes sollte jedoch vermieden werden (z.\,B. ein Abschnitt 3.4.2.5.3).

\subsection{Tabellen und Abbildungen}
Tabellen und Abbildungen sind sogenannte \textit{Floating Objects}, d.\,h. \LaTeX\ setzt diese Objekte an Positionen, die satztechnisch geeignet sind. Daher kann es vorkommen, dass Tabellen oder Abbildungen auf einer anderen Seite erscheinen, die dann referenziert werden müssen. Hier ein Beispiel dafür:

\LaTeX~ kümmert sich darum, wo die Abbildungen gesetzt werden und passt den Text der Referenz entsprechend an. Soll nur die Nummerierung in den Text geschrieben werden, dann kann der Befehl \texttt{\textbackslash ref} verwendet werden. Abbildungen sollten -- falls möglich -- als Vektor-PDF eingebunden werden, da die diese dann beliebig skalieren können.

\lipsum[1]

\begin{table}
	\centering
	\begin{tabular}{p{3cm}crl}
		\textbf{Spalte 1} & \textbf{Spalte 2} & \textbf{Spalte 3} & \textbf{Spalte 4}\\\toprule
		Zeile 1 Spalte 1 &  Zeile 1 Spalte 2 & Zeile 1 Spalte 3 & Zeile 1 Spalte 4\\
		Zeile 2 Spalte 2 &  Zeile 2 Spalte 2 & Zeile 2 Spalte 3 & Zeile 2 Spalte 4\\\midrule
		Zeile 3 Spalte 1 &  Zeile 3 Spalte 2 & Zeile 3 Spalte 3 & Zeile 3 Spalte 4\\
		Zeile 4 Spalte 1 &  Zeile 4 Spalte 2 & Zeile 4 Spalte 3 & Zeile 4 Spalte 4\\\bottomrule
	\end{tabular}
	\caption[Testtabelle]{Testtabelle}
	\label{tab:tabelle1}
\end{table}
\lipsum[1-2]

\begin{figure}
	\centering
	\includegraphics[width=\textwidth,height=\textheight,keepaspectratio]{dhbw_logo.png}
	\captionsetup{format=hang}
	\caption[Optionaler Kurztitel für das Abbildunggsverzeichnis]{Demo-Abbildung, um zu verdeutlichen, wie gleitende Objekte gesetzt werden und wie entsprechend die Quelle zitiert wird. \\Quelle: \cite{imgmj}}
	\label{fig:test}
\end{figure}

\subsection{Listings}

Das Einbinden eines Listings mit der entsprechenden Umgebung ist auch kein Problem, wie man in Listing \ref{lst:helloworld} sehen kann. Schauen Sie sich hierzu das \texttt{listings}-Paket an!

\begin{lstlisting}[language=Java, caption={Hello World!}, label={lst:helloworld}]
public static void main(String args[]) {
   System.out.println("Hello World!");
}
\end{lstlisting}


\subsection{Mathematische Formeln}
Auch mathematische Ausdrücke können mit \LaTeX~ sehr gut gesetzt werden, wie man anhand der Gleichung \ref{eqn:e2} sehen kann -- konsultieren Sie hierzu bitte entsprechende Dokumentationen, die Online zur Verfügung stehen.

\begin{equation}
f(x)=x^2
\label{eqn:e2}
\end{equation}

\begin{figure}
\begin{equation}
f(x)=x^2
\end{equation}
\caption{Quadratic Function}
\end{figure}


\subsection{Algorithmen}
Algorithmen können als Pseudocodes dargestellt und referenziert werden, wie z.\,B. in Algorithmus \ref{alg:euclid} -- sogar bis auf Zeilennummern
(siehe die \texttt{while}-Anweisung in Zeile \ref{alg:euclid:while}). Schauen Sie sich hierzu bitte das Paket \texttt{algorithmicx} an.


\begin{algorithm}
\begin{algorithmic}[1]
\Procedure{Euclid}{$a,b$}\Comment{The g.c.d. of a and b}
   \State $r\gets a\bmod b$
   \While{$r\not=0$}\Comment{We have the answer if r is 0} \label{alg:euclid:while}
      \State $a\gets b$
      \State $b\gets r$
      \State $r\gets a\bmod b$
   \EndWhile\label{euclidendwhile}
   \State \textbf{return} $b$\Comment{The gcd is b}
\EndProcedure
\end{algorithmic}
\caption{Euklid'scher Algorithmus}\label{alg:euclid}
\end{algorithm}


\subsection{Glossar}
Der \gls{glossar} wird automatisch erzeugt und zeigt alle referenzierten Glossareinträge an. Um einen neuen Glossareintrag zu erstellen verwendet man den Befehl\\ \texttt{\textbackslash newglossaryentry\{<label>\}\{<settings>\}}.

Ein erstellter Glossareintrag kann mit folgenden befehlen referenziert werden:
\begin{table}[h!]
	\centering
	\caption[Glossar Befehle]{\label{tab:glossarTabelle}Glossar Befehle}
	\begin{tabular}{lp{10cm}}
		\textbf{Befehl} & \textbf{Effekt}\\\hline
		\texttt{\textbackslash gls\{<label>\}} & Referenziert den angegebenen Begriff und erstellt einen Link zum Glossareintrag.\\
		\texttt{\textbackslash glspl\{<label>\}} & Referenziert den Plural des angegebenen Begriffs falls vorhanden und erstellt einen Link zum Glossareintrag.\\
		\texttt{\textbackslash Gls\{<label>\}} & Referenziert den angegebenen Begriff mit großem Anfangsbuchstaben und erstellt einen Link zum Glossareintrag\\
		\texttt{\textbackslash Glspl\{<label>\}} & Referenziert den Plural des angegebenen Begriffs mit großem Anfangsbuchstaben falls vorhanden und erstellt einen Link zum Glossareintrag.\\\hline
	\end{tabular}
\end{table}


%	Literaturverzeichnis
\ihead{} % Neue Header-Definition
% \printbibliography
\printbibheading
% \printbibliography[nottype=online, nottype=misc, notkeyword=image, notkeyword=reference, heading=subbibliography, title={Publication Sources}]
% \printbibliography[type=online, notkeyword=image, notkeyword=reference, heading=subbibliography, title={Online Sources}]
% \printbibliography[type=misc, keyword=image, heading=subbibliography, title={Image Sources}]
% \printbibliography[type=misc, keyword=reference, heading=subbibliography, title={Reference Sources}]
\printbibliography[notkeyword=image, heading=subbibliography, title={Sources}]
\printbibliography[type=misc, keyword=image, heading=subbibliography, title={Image Sources}]
\clearpage


% 	Glossaries
\printglossaries
\clearpage


%----------------------------------------
% Anhang
% Hier können alle Anhänge als einzelne Datei eingefügt werden.
% Die Anhänge werden gesondert alphabethisch benannt (A Anhang 1, B Anhang 2, usw)
%----------------------------------------
\appendix
\ihead{\appendixname~\thechapter} % Neue Header-Definition

% !TEX root = ../main.tex

\chapter{Interview \enquote{KWS~Computersysteme}}
\label{chap:appendix:interview:kws}
\begin{otherlanguage*}{ngerman}
\begin{linenumbers*}
\begin{description}[leftmargin=4cm, labelwidth=4cm]
    \item[Interviewer] \noindent \lipsum[1]
    \item[Partner] \noindent \lipsum[2]
    \item[Interviewer] \noindent \lipsum[1-2]
    \item[Partner] \noindent \lipsum[4]
\end{description}
\end{linenumbers*}
\end{otherlanguage*}

\end{document}