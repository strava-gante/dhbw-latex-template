% !TEX root = ../main.tex

%-------------------
% 		HYPERREF
%-------------------

\usepackage{hyperref}
\hypersetup{
    colorlinks = true,
    % allcolors = [rgb]{0,0,0} % black
    allcolors = [rgb]{0,0,0.4} % blue
}

% URLs im Fließtext
\newcommand{\urlinline}[1]{\textsf{\footnotesize{\url{{#1}}}}}


%		TITLING
%-----------------------------------
\makeatletter
\newcommand*{\thetitle}{\@title}
\newcommand*{\theauthor}{\@author}
\newcommand*{\thedate}{\@date}
\newcommand*{\place}[1]{\newcommand{\theplace}{#1}}
\newcommand*{\thesis}[1]{\newcommand{\thethesis}{#1}}
\newcommand*{\university}[1]{\newcommand{\theuniversity}{#1}}
\newcommand*{\courseofstudies}[1]{\newcommand{\thecourseofstudies}{#1}}
\newcommand*{\timeperiod}[1]{\newcommand{\thetimeperiod}{#1}}
\newcommand*{\studentnr}[1]{\newcommand{\thestudentnr}{#1}}
\newcommand*{\course}[1]{\newcommand{\thecourse}{#1}}
\newcommand*{\company}[1]{\newcommand{\thecompany}{#1}}
\newcommand*{\supervisor}[1]{\newcommand{\thesupervisor}{#1}}
\newcommand*{\scientificsupervisor}[1]{\newcommand{\thescientificsupervisor}{#1}}
\makeatother

% PDF settings
\hypersetup{pdftitle={\thetitle}}
\hypersetup{pdfauthor={\theauthor}}


%-------------------
% 		List Styles
%-------------------
\usepackage{enumitem}
\usepackage{lineno}


%-------------------------
%		SCHRIFTART
%-------------------------
% Entweder Latin Modern oder Times / Helvetica
\usepackage{lmodern} %Latin modern font
%\usepackage{mathptmx}  %Helvetica / Times New Roman fonts (2 lines)
%\usepackage[scaled=.92]{helvet} %Helvetica / Times New Roman fonts (2 lines)


%-----------------------------------
%		SCHRIFT UND ENCODING
%-----------------------------------
\usepackage[T1]{fontenc}

\usepackage{eurosym} % euro symbol
\DeclareRobustCommand{\officialeuro}{%
  \ifmmode\expandafter\text\fi
  {\fontencoding{U}\fontfamily{eurosym}\selectfont e}}
  
\usepackage[utf8]{inputenc}
\DeclareUnicodeCharacter{20AC}{\euro} % Euro symbol character

\usepackage{setspace}

% Common abbreviation's
\usepackage{xspace}
\AtBeginDocument{
    \newcommand*{\eg}{e.g.\@\xspace}
    \newcommand*{\ie}{i.e.\@\xspace}
    \newcommand*{\Eg}{E.g.\@\xspace}
    \newcommand*{\Ie}{I.e.\@\xspace}
    \newcommand*{\cf}{cf.\@\xspace}
    \newcommand*{\Cf}{Cf.\@\xspace}
}


%---------------------------
%		BERECHNUNGEN
%---------------------------
\usepackage{calc} % Used for extra space below footsepline


%-------------------------
%		TABLES
%-------------------------
\usepackage[table,xcdraw]{xcolor}


%---------------------------------
%		SPRACHEINSTELLUNGEN
%---------------------------------
% Voreinstellungen für Deutsch und Englisch. Die nicht verwendete Sprache ist auszukommentieren.
% DEUTSCH
%\usepackage[ngerman]{babel}
%\usepackage[german=quotes]{csquotes}

% ENGLISH
%\usepackage[ngerman]{babel}
%\usepackage[german=quotes]{csquotes}

%ENGLISH & DEUTSCH (default english)
\usepackage[english, ngerman]{babel}
\usepackage[style=american]{csquotes} % Richtiges Setzen der Anführungszeichen mit \enquote{}
\AtBeginDocument{
    \selectlanguage{english}
}


%----------------------------
%		BIBLIOGRAFIE
%----------------------------
\AtBeginDocument{
    \setcounter{biburllcpenalty}{7000} % break urls
    \setcounter{biburlucpenalty}{8000} % break urls
}
% Voreinstellungen für Fußnotenzitate (Autor-Jahr), IEEE-Standard, Alphabetic-Stil und Havard-Stil. Die nicht verwendeten Stile müssen auskommentiert werden

%\usepackage[backend=biber, autocite=footnote, style=authoryear, dashed=false]{biblatex}                % Fußnotenzitate
% \usepackage[backend=biber, autocite=inline, style=ieee]{biblatex}                                     % IEEE-Stil
% \usepackage[backend=biber, autocite=inline, style=alphabetic]{biblatex}                               % Alphabetic-Stil
% \usepackage[backend=biber, autocite=inline, style=authoryear, dashed=false]{biblatex}		            % Harvard-Stil
\usepackage[backend=biber, autocite=inline, bibstyle=authoryear, citestyle=apa, dashed=false]{biblatex} % APA-Stil


% Fußnotenzitate mit YYYY-MM-DD in Bibliographie
% \usepackage[backend=biber, autocite=footnote, style=authoryear, dashed=false, urldate=edtf, date=edtf, seconds=true]{biblatex}

\DefineBibliographyStrings{ngerman}{  %Change u.a. to et al. (german only!)
	andothers = {{et\,al\adddot}},
}

\setlength{\bibparsep}{\parskip}		%add some space between biblatex entries in the bibliography
\addbibresource{adds/bibliography.bib}	%Add file bibliography.bib as biblatex resource


%----------------------
%		ACRONYME
%----------------------
%%%
%%% WICHTIG: Installieren Sie das neueste Acronyms-Paket!!!
%%%
\makeatletter
\usepackage[]{acronym}
\@ifpackagelater{acronym}{2015/03/20}
  {%
    \renewcommand*{\aclabelfont}[1]{\textbf{\textsf{\acsfont{#1}}}}
  }%
  {%
  }%
\makeatother


%--------------------
%		GLOSSAR
%--------------------
%\usepackage[toc]{glossaries}					% für Seitenreferenzen im Glossar
\usepackage[toc, nonumberlist]{glossaries}		% ohne Seitenreferenzen im Glossar


%--------------------
%		NUMBERING FIGURES/TABLES/FOOTNOTES/TOC/CHAPTERS
%--------------------
\usepackage{chngcntr}
\counterwithout{footnote}{chapter} % Zum Zählen der Fußnoten über Kapitel hinaus
% \counterwithin{figure}{section} % Zum Zählen der Bilder innerhalb von Sections
% \counterwithin{table}{section} % Zum Zählen der Tabellen innerhalb von Sections
\setcounter{tocdepth}{2} % TODO adapt if toc too long
\setcounter{secnumdepth}{2} % Numbering for subsubsections


%-----------------------------------
%		REFERENCES & CITING
%-----------------------------------
% \usepackage{letltxmacro}
% \AtBeginDocument{
%     \LetLtxMacro{\originalref}{\ref}
%     \renewcommand{\ref}[1]{(\cf~\originalref{#1}, p.~\pageref{#1})}
%     \newcommand{\imgref}[1]{(\cf~figure~\originalref{#1}, p.~\pageref{#1})}
% }


%-------------------------------
%		ZUSÄTZLICHE PAKETE
%-------------------------------
\usepackage{lipsum}				 % Blindtext
\setlipsumdefault{1-4}

\usepackage[activate]{microtype} % fix overfull hbox
\usepackage[pdftex]{graphicx}    % verschiene Bildformate einbinden
\usepackage{pdfpages}            % PDF einbinden
\usepackage[format=hang, justification=raggedright, margin=1cm, singlelinecheck=false]{caption}			% schönere Überschriften

\usepackage{booktabs}			% bessere Tabs
\usepackage{array}

% bessere Tabellen
\usepackage{multirow}
\setlength{\tabcolsep}{10pt}
\renewcommand{\arraystretch}{1.5}
\newcolumntype{P}[1]{>{\raggedright\arraybackslash}p{#1}}

% resource folder einbinden
\graphicspath{ {resources/} }


%--------------------------
%		Tikz diagram library
%--------------------------
\usepackage{tikz}
\usetikzlibrary{arrows,decorations.pathmorphing,backgrounds,fit,positioning,shapes.symbols,chains}

%--------------------------
%		TpX used packages
%--------------------------
\usepackage{color}
\DeclareGraphicsExtensions{.pdf,.png,.jpg,.jpeg,.mps}
\usepackage{pgf}
\usepackage{epic,bez123}
\usepackage{floatflt}% package for floatingfigure environment
\usepackage{wrapfig}% package for wrapfigure environment

%-------------------------
%		ALGORITHMEN
%-------------------------
\usepackage{algorithm}
\usepackage{algpseudocode}
\renewcommand{\listalgorithmname}{List of Algorithms}
\floatname{algorithm}{algorithm}

%------------------------------------
%		KOPFZEILE / FUßZEILE
%------------------------------------
%	   ACHTUNG! Einige einstellungen werden in master.tex erneut verändert
\RequirePackage[automark,headsepline,footsepline]{scrpage2}
\pagestyle{scrheadings}
\renewcommand*{\pnumfont}{\upshape\sffamily}
\renewcommand*{\headfont}{\upshape\sffamily}
\renewcommand*{\footfont}{\upshape\sffamily}
\renewcommand{\chaptermarkformat}{}

\clearscrheadfoot

% using cfoot centers the header
% use ofoot for outer side, especially with documentclass[twoside]
\cfoot[\rule{0pt}{\ht\strutbox+\dp\strutbox}\pagemark]{\rule{0pt}{\ht\strutbox+\dp\strutbox}\pagemark}

\ohead{\headmark}

%-----------------------------------
%		Fix Headheight
%-----------------------------------
\setlength{\headheight}{1.1\baselineskip}

